%
% tobysheaders.tex
% Header file for all documents created by Toby Dylan Hocking
% <tobob@berkeley.edu> in LaTeX
% No commands are to be executed in this header file
% Commands, lengths, etc. may only be created and renewed
%

%%
%% Standard headers
%%

% Personal Website
\newcommand{\homepage}{{\tt http://www.ocf.berkeley.edu/\~{}tdhock}}
% rlmm website
\newcommand{\rlmmhomepage}{{\tt \homepage/rlmm}}

% Email
\newcommand{\email}{{\tt tobob@berkeley.edu}}

\usepackage{graphicx}
%
% \newfig
% #1: width
% #2: filename
% #3: label
% #4: caption
%
\newcommand{\newfig}[4][\textwidth]{
	\begin{figure}
		\begin{center}
			\includegraphics[width=#1]{#2}
		\end{center}
		\caption{\label{#3}#4}
	\end{figure}
}
%
% \myfig
% #1: width
% #2: filename = label
% #3: caption
%
\newcommand{\myfig}[3][\textwidth]{
	\begin{figure}
		\begin{center}
			\includegraphics[width=#1]{#2}
		\end{center}
		\caption{\label{#2}#3}
	\end{figure}
}
%
% \tabref and \figref
% For referring to figures and tables
%
\newcommand{\tabref}[1]{Table~\ref{#1}}
\newcommand{\figref}[1]{Figure~\ref{#1}}

\newcommand{\blank}{\underline{\ \ \ \ \ \ \ \ \ \ }}

% \ti(tem) makes a \tt item in a description environment
\newcommand{\ti}[2]{\item[\fbox{\tt #1}]#2}
%
% Pretty sweet "automatic" glossary and index system
% 1. Use \boldword{foo}{bar} to print the bold word foo and have
%    its description, bar, appear in the glossary (foo will also
%    appear in the index)
% 2. Use \pindex to print a word and add it to the index
% 3. Include the commands \makeindex and \makeglossary in the
%    header of the file
% 4. At the end of the file, include a \glossaryandindex, which
%    reads the .gls and .ind files in step 9
% 5. latex (builds .ind and .gls files)
% 6. bibtex
% 7. makeindex file.ind
% 8. makeindex file.gls (Make sure you have a glo.ist file)
% 9. latex (builds .aux file)
% 10.latex
%
\newcommand{\pindex}[1]{#1\index{#1}}
% \boldword prints the word in bold and adds it
%  to the glossary, with the second argument as
%  the definition
%1. Key word
%2. glossary definition
\newcommand{\boldword}[2]{{\bf #1\/}\index{#1}\glossary{[#1] #2}}

% \glossaryentry's generated by \boldword
%  should work 
% Remember: \makeglossary generates \glossaryentry commands
\newcommand{\glossaryentry}[2]{\item[#1 (#2)}
\newenvironment{theglossary}{\bd}{\ed}

% \glossaryandindex
%  the idea here is to pass the file's base name as an argument
%  1. Comment \glossaryandindex and run LaTeX
%  2. Run CMacTeX > Apps > make index... sty file `glo.ist'
%  3. Uncomment \glossaryandindex and comment \makeglossary
%  4. Run LaTeX
\newcommand{\glossaryandindex}[1]{
\appendix
\begin{chapter}{Glossary}
\input{#1.gls}
\end{chapter}
\input{#1.ind}}

\newcommand{\newtab}{\begin{center}\begin{tabular}}
\newcommand{\et}{\end{tabular}\end{center}}

\newcommand{\be}{\begin{enumerate}}
\newcommand{\ee}{\end{enumerate}}
\newcommand{\bi}{\begin{itemize}}
\newcommand{\ei}{\end{itemize}}
\newcommand{\bd}{\begin{description}}
\newcommand{\ed}{\end{description}}
\newcommand{\bs}{\begin{section}}
\newcommand{\es}{\end{section}}
\newcommand{\bss}{\begin{subsection}}
\newcommand{\ess}{\end{subsection}}

\newcommand{\resume}{r\'{e}sum\'{e}}
\newcommand{\Resume}{R\'{e}sum\'{e}}
\newcommand{\etal}{{\em et al.\/}}
%
% \defmargins
% bottom,left,top,right
%
\newcommand{\defmargins}[4]{

	\setlength{\topmargin}{0in}
	\addtolength{\topmargin}{#3}
	\addtolength{\topmargin}{-1in}
	
	\setlength{\oddsidemargin}{0in}
	\addtolength{\oddsidemargin}{#2}
	\addtolength{\oddsidemargin}{-1in}
	
	\setlength{\textheight}{10in}
	\addtolength{\textheight}{-#1}
	\addtolength{\textheight}{-\topmargin}
	
	\setlength{\textwidth}{7.5in}
	\addtolength{\textwidth}{-#4}
	\addtolength{\textwidth}{-\oddsidemargin}
}

\newcommand{\goodmargins}{\defmargins{2in}{1in}{.5in}{1in}}
\newcommand{\doublespace}{\renewcommand{\baselinestretch}{2}}
\newcommand{\singlespace}{\renewcommand{\baselinestretch}{1}}

% Author(s), Journal Volume, Page (Year)
\newcommand{\scifoot}[5]{\footnote{#1, {\em #2}\/ {\bf #3}, #4 (#5).}}

% Strike thru format
% Draws a .01inch line 1/2 the height of #1 over #1
\newlength{\strikelength}
\newlength{\strikeheight}
\newcommand{\strikethru}[1]
{\settowidth{\strikelength}{#1}\settoheight{\strikeheight}{#1}#1\hspace{-1\strikelength}\rule[0.5\strikeheight]{\strikelength}{.01in}}

% kantar logo
\newcommand{\kantar}{$\cal K$\hspace{-.7ex}\raisebox{-.2ex}{A}\raisebox{-.2ex}{\hspace{-.6ex}N}\raisebox{.35ex}{\hspace{-.4ex}T}\hspace{-1ex}\raisebox{-.2ex}{A}\hspace{-.5ex}R}

% Letters

\newcommand{\formletter}[5]
{
	
	\begin{letter}{#1 #2 \\ #3}
	
	\opening{Dear #1,}
	
	#4
	
	\closing{Sincerely,}
	
	\ps{#5}
	
	\end{letter}
	
	\begin{flushleft}
	Toby Dylan Hocking \\
	2340 LeConte Avenue \#203 \\
	Berkeley, CA 94709 \\
	\end{flushleft}
	
	\begin{center}
	#1 #2 \\ #3
	\end{center}
}

%%
%% History Headers
%%

% Names
\newcommand{\schro}{Schr\"{o}dinger}

%%
%% CS Headers
%%

% Setting off code

\usepackage{verbatim}
\newenvironment{code}{\begin{quote}\tt\small}{\end{quote}}
\newcommand{\codefile}[1]{

\begin{flushleft}
\underline{#1}
\nopagebreak
\begin{code}
\verbatiminput{#1}
\end{code}
\end{flushleft}

}

\newcommand{\tq}[1]{
\begin{flushleft}
{\tt\small
#1
}
\end{flushleft}
}

% Describing a function
% the idea is to use `\argu' as #2 arguments to `\function'

\newcommand{\argu}[2]{\item {\tt #1}: #2}
%1.name
%2.description

\newcommand{\function}[5]{
\begin{figure}
	\bd
	\item[{\large #1}] \hrulefill
	\item[Arguments:]
		\be \item[]
		#2
		\ee
	\item[Return Value:] #3
	\item[Description:] #4
	\item[Examples:]
	{\tt
	\begin{tabbing}
		\\
		#5
		\end{tabbing}
	}
	\item \hrulefill
	\ed
\end{figure}
}
%1.name
%2.arguments
%3.return
%4.description
%5.example

%%
%% MCB Headers
%%
%
% \genophenotab
% #1: Genotypes and Phenotypes (tabular)
% #2: Table label
% #3: Table caption
%
\newcommand{\genophenotab}[3]{
	\begin{table}
		\newtab{cl}
			Genotype & Phenotype \\
			\hline
			#1
		\et
		\caption{\label{#2}#3}
	\end{table}
}

\newcommand{\pr}{$^{\prime}$}
\newcommand{\fivethreeprime}{5\pr$\rightarrow$3\pr}
\newcommand{\dg}{$^{\circ}$}

% Enzyme Abbreviations
\newcommand{\hind}{{\rm{\it Hin}dIII}}
\newcommand{\xba}{{\rm{\it Xba}I}}
\newcommand{\bam}{{\rm{\it Bam}HI}}
\newcommand{\eco}{{\rm{\it Eco}RI}}
\newcommand{\pst}{{\rm{\it Pst}I}}
\newcommand{\sca}{{\rm{\it Sca}I}}
\newcommand{\xho}{{\rm{\it Xho}I}}

% Molecular abbreviations
\newcommand{\RNAi}{RNA$_{\text{i}}$}

% Punnett Squares

\newcommand{\homocross}[3]
{
	{\tt
		\newtab{|r|c|}
		\hline
		{\bf #3} & #1 \\
		\hline
		#2 & #1#2 \\
		\hline
		\et
	}
}

\newcommand{\monohybrid}[5]
{
	{\tt
		\newtab{|r|cc|}
		\hline
   {\bf #5}&  #1  & #2 \\
		 \hline
		#3 & #1#3 & #2#3 \\
		#4 & #1#4 & #2#4 \\
		\hline
		\et
	}
}

\newcommand{\dihybrid}[9]
{
	{\tt
		\newtab{|r|cccc|}
		\hline
	 {\bf #9}& #1#3	    &   #1#4   &	#2#3  &    #2#4  \\
		\hline
		#5#7 & #1#5#3#7 & #1#5#7#4 & #1#5#3#7 & #5#2#7#4 \\
		#5#8 & #1#5#3#8 & #1#5#4#8 & #5#2#3#8 & #5#2#4#8 \\
		#6#7 & #1#6#7#3 & #1#6#7#4 & #2#6#3#7 & #2#6#7#4 \\
		#6#8 & #1#6#3#8 & #1#6#4#8 & #2#6#3#8 & #2#6#4#8 \\
		\hline
		\et
	}
}

% Cross Diagrams

% The idea is to use \cross as the argument to \ctab
\newcommand{\ctab}[2]
{
	\newtab{ccccccl}
	\hline
	#1 \\
	\hline
	#2 \\
	\hline
	\et
}
%
% \cross
% Use two \cross commands as arguments for \ctab
% #1: parental generation abbrev
% #2: p genotype 1
% #3: p genotype 2
% #4: progeny genotypes \prog
% #5: progeny generation
%
\newcommand{\cross}[5]
{{\bf #1} & #2 & $\times$ & #3 & $\Rightarrow$ &
\begin{tabular}{rl}
#4
\end{tabular}
& #5}

% Use as many \prog commands for the fourth argument of \cross as there are progeny
% #1: Description of progeny
% #2: Number of progeny
\newcommand{\prog}[2]
{#2 & #1 \\}

% Drosophila
% Including marvosym blocks \Rightarrow somehow
%\usepackage{marvosym}
\newcommand{\wt}{$^{+}$}
\newcommand{\dros}{{\em Drosophila\/}}

% Bacteria
\newcommand{\ecoli}{{\em E. coli\/}}
\newcommand{\cistrans}{{\em cis-trans\/}}
\newcommand{\trans}{{\em trans\/}}
\newcommand{\cis}{{\em cis\/}}

% Yeast
\newcommand{\scer}{{\em S. cerevisiae\/}}
\newcommand{\neur}{{\em Neurospora crassa\/}}

%%
%% Math Headers
%%

\newcommand{\dismat}[3]{
$$
#2 = 
\left[
\begin{array}{#1}
#3
\end{array}
\right]
$$
}

\usepackage{amsmath,amssymb}
%
% \quadform
% a,b,c
%
\newcommand{\quadform}[3]{
\frac{ -(#2) \pm \sqrt{(#2)^2 - (4\times#1\times#3)}}{2\times#1}}

\newcommand{\iid}{\stackrel{\rm iid}}
%
% \relstack
% opposite of \stackrel
%
\newcommand{\relstack}[2]{\stackrel{\textstyle #1}{\scriptstyle #2}}

\newcommand{\mse}{\operatorname{MSE}}
\newcommand{\Span}{\operatorname{Span}}
\newcommand{\rank}{\operatorname{rank}}
\newcommand{\nullity}{\operatorname{nullity}}
\newcommand{\tr}{\operatorname{tr}}
\newcommand{\Var}{\operatorname{Var}}
\newcommand{\Cov}{\operatorname{Cov}}
\newcommand{\med}{\operatorname{Median}}
\newcommand{\pois}{\operatorname{Poisson}}
\newcommand{\Bin}{\operatorname{Binomial}}
\newcommand{\lik}{\operatorname{Lik}}

\newcommand{\LL}{\lambda}
\newcommand{\lbar}{\bar{\lambda}}
\newcommand{\lo}{\lambda_0}
\newcommand{\lhat}{\hat{\lambda}}

\newcommand{\xbar}{\bar{X}}
\newcommand{\xbars}{\bar{X}_{\text{s}}}

\newcommand{\ybar}{\bar{Y}}
\newcommand{\yhat}{\hat{Y}}

\newcommand{\mubar}{\bar{\mu}}
\newcommand{\muhat}{\hat{\mu}}

\newcommand{\sigbar}{\bar{\sigma}}
\newcommand{\sigxbar}{\sigma_{\bar{X}}}
\newcommand{\shat}{\hat{\sigma}}
\newcommand{\shatx}{\hat{\sigma}_{\bar{X}}}
\newcommand{\shatp}{\sigma_{\hat{p}}}

\newcommand{\that}{\hat{\theta}}
\newcommand{\tho}{\theta_0}
\newcommand{\sthat}{s_{\hat{\theta}}}
\newcommand{\sphat}{s_{\hat{p}}}
\newcommand{\sxbar}{s_{\bar{X}}}

\newcommand{\phat}{\hat{p}}
\newcommand{\xhat}{\hat{X}}
\newcommand{\phati}{\hat{p}_i}

\newcommand{\gammahat}{\hat{\gamma}}
\newcommand{\alphahat}{\hat{\alpha}}
\newcommand{\Gammahat}{\hat{\Gamma}}

\newcommand{\rhohat}{\hat{\rho}}

\newcommand{\ept}{\epsilon_t}
\newcommand{\dth}{\delta_{t+h}}
\newcommand{\dhat}{\hat{\Delta}}
\newcommand{\fpc}{\left( 1 - \frac{n-1}{N-1} \right)}

\newcommand{\RR}{\mathbb R}
\newcommand{\ZZ}{\mathbb Z}
\newcommand{\CC}{\mathbb C}
\newcommand{\NN}{\mathbb N}
\newcommand{\QQ}{\mathbb Q}

\newcommand{\cals}{{\cal S}}
\newcommand{\call}{{\cal L}}

\newcommand{\cont}{{\sf C}}
\newcommand{\diff}{{\sf D}}
\newcommand{\VV}{{\sf V}}
\newcommand{\UU}{{\sf U}}
\newcommand{\FF}{{\sf F}}
\newcommand{\TT}{{\sf T}}
\newcommand{\PP}{{\sf P}}
\newcommand{\MM}{{\sf M}}
\newcommand{\WW}{{\sf W}}

\newcommand{\del}{\partial}
\newcommand{\prm}{^{\prime}}

\newcommand{\beqn}{\begin{equation}\label}
\newcommand{\eeqn}{\end{equation}}
\newcommand{\eref}[1]{\text{(Eqn. \ref{#1})}}
\newcommand{\teref}[1]{Equation \ref{#1}}

\newtheorem{theorem}{Theorem}
\newtheorem{defn}{Definition}
\newtheorem{prob}{Problem}
\newtheorem{lemma}{Lemma}
\newtheorem{ques}{Question}
\newtheorem{claim}{Claim}
\newtheorem{cor}{Corollary}

\newcommand{\hoha}[2]{
	\begin{eqnarray*}
	H_0 &:& #1 \\
	H_A &:& #2
	\end{eqnarray*}
}