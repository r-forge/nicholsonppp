\documentclass[a4paper,12pt]{article}
\usepackage{graphicx}
\usepackage{hyperref}
\usepackage[all]{xy}
\usepackage{amsmath,amssymb}

\title{Identifying genetic loci under selective pressure using a
  posterior predictive p-value classifier} \author{Toby Dylan Hocking}

\begin{document}

\maketitle
\tableofcontents
\newpage
\listoffigures
\newpage

\newcommand{\fig}[3][1]{
  \begin{figure}[htp]
    \begin{center}          %linewidth
    \includegraphics[width=#1\linewidth]{#2}
    \end{center}
    \caption{#3\label{#2}}
  \end{figure}
}
% \brat = brace matrix
\newcommand{\brat}[2]{
  \left[
    \begin{array}{#1}
      #2
    \end{array}
    \right]
}
\newcommand{\aij}{\alpha_{ij}(t)}
\newcommand{\aijs}{\alpha_{ij}^*(t)}
\newcommand{\wij}[1]{w_{ij}^{\text{#1}}}
\newcommand{\etal}{\emph{et al.}}
\newcommand{\RR}{\mathbb R}
\newcommand{\Bin}{\operatorname{Binomial}}

\section{Introduction}

The recent explosion of molecular marker data in animal populations
from technologies such as Single Nucleotide Polymorphism (SNP) assays
opens new avenues of research for population genetics. These data
allow testing of many aspects of our current models of population
genetics, such as the evolutionary status of these markers relative to
selective pressures. These data are usually investigated by either
examining summary statistics and empirical distributions, or by using
model-based approaches. We are more concerned with model-based
approaches, with emphasis on detecting departures from the model.

In this study, we are interested in is the estimation of genetic
differentiation between populations, and establishing methods for
determining which markers and genomic regions have been under
selective pressure. Genomic areas under selection are areas with
probable functional significance, thus the goal is to develop methods
we can use to identify functional genes and augment the current state
of functional annotations for domestic animal genomes.

In the last several decades, the statistical tools of biologists and
geneticists have evolved considerably. In particular, modern computers
and stochastic methods such as Markov Chain Monte Carlo (MCMC) allow
for estimation of the posterior distribution of parameters of Bayesian
models of evolutionary systems. In this work, we investigate one such
Bayesian model used to describe evolution of domestic animal
populations, and extend it with a classifier for markers under
selection.

The model of domestic animal evolution that we consider in this work
is the hierarchical Bayesian model of Nicholson \etal\
\cite{nicholson}, hereafter referred to as the Nicholson model. This
model assumes that a single population gives rise to several
subpopulations, which branch off at the same time, and begin
independently evolving. The Nicholson model assumes all loci are
affected only by genetic drift (not selection), and attempts to
measure population differentiation in a manner which is analogous to
the classical $F_{ST}$ of population genetics. In the process the
model also yields estimates of ancestral allele frequency.

The new idea put forth in this work is a classifier for markers under
selection, based on loci which do not fit well into the pure-drift
Nicholson model. To quantify the probability that a locus fits the
pure-drift model, we use Posterior Predictive P-values, or PPP-values
\cite{pppvalues}. Essentially these PPP-values are the Bayesian
analogue of the usual frequentist P-values, which indicate departures
from the model hypotheses. The Nicholson model was designed for, and
accurately models, independently evolving populations under genetic
drift. However, loci under selection in addition to genetic drift
represent departures from the model hypotheses. Thus we use PPP-values
estimates from the model to identify these aberrent loci.

Furthermore, to test the robustness of our classifier under different
evolutionary conditions, we test it using extensive simulation of
evolution by genetic drift and selection. The simulator has been
implemented using the R programming language \cite{R}. The model
fitting has been implemented using compiled FORTRAN code dynamically
linked to R. The simulations, analyses, and graphics discussed in this
article can be reproduced by using the code published in the R package
\texttt{nicholsonppp} on R-Forge \cite{R-Forge}:

 \url{http://nicholsonppp.r-forge.r-project.org/}

\section{Methods}

\subsection{Simulating genetic drift and selection}

To simulate selection and genetic drift in several independent
populations, we use a modified version of the simulator in
\cite{Beaumont-Balding}. We assume $L$ independent loci, and $P$
independent populations, evolving over $T$ generations.

To model SNP data, each locus has only 2 possible alleles, thus we
denote the allele frequency for locus $i$ in population $j$ at
generation $t$ as $\aij$, with $0\leq \aij\leq 1$. To assign ancestral
allele frequencies $\pi_i=\alpha_{i,1}(1)=...=\alpha_{i,P}(1)$, we
draw from a truncated $\beta(0.7,0.7)$ distribution
(\autoref{beta}). That is, for all $i$,
$$P(\pi_i<0.05)=P(\pi_i>0.95)=0$$
and if $Z\sim\beta(0.7,0.7)$,
$$P(\pi_i=0.05)=P(\pi_i=0.95)=P(Z<0.05)$$

We chose to truncate the distribution of initial allele frequencies so
as to reduce the number of loci which are ``fixed'' with allele
frequency 0 or 1 at the end of the simulation.

This choice is motivated by population genetics, which gives us the
result that at equilibrium, under mutation and genetic drift, the
distribution of allele frequencies approximately follow a
$\beta(4N\mu,4N\mu)$ distribution, where $N$ is the effective
population size and $\mu$ is the mutation rate \cite{Wright}. We
also considered using a truncated $\beta(1,1)$ distribution, which is
the same as the $U[0,1]$ distribution, but no noticeable difference
was observed.

\fig[0.75]{beta}{Distribution of ancestral allele frequencies in our
  simulations follow a truncated $\beta(0.7,0.7)$ distribution.}

If we assign the colors blue and red to the 2 alleles, and we define
$\aij$ as the blue allele frequency for locus $i$ in population $j$ at
generation $t$, then $1-\aij$ is the red allele frequency. The color
of the allele will determine its fate under selection in populations
which each have a background color of either red or blue. Population
color is chosen at random, independently for each locus, at the
beginning of the simulation, with probability 0.4 for each of red or
blue, and 0.2 for neutral populations where neither allele is favored.

The idea is to simulate the fact that some alleles are favored in some
environments, while disfavored in others. Thus, under positive
selection, red alleles will be favored in red populations, and
disfavored in blue populations (vice versa for blue alleles). Under
balancing selection, it is advantageous to have both a blue and red
allele. This simulates the heterozygote advantage, a phenomenon
classically observed in the gene that controls malaria resistence and
sickle-cell anemia affectation.

The allele frequency changes in each generation via
2 mechanisms: drift and selection. Drift introduces some random
variability up or down in allele frequency, independent of population
color:
$$\aijs \sim \operatorname{Bin}(N_{ij},\alpha_{ij}(t-1))$$
The effect of drift grows more important relative to selection as
population size $N_{ij}$ diminishes.

Then to update the allele frequency for selection, we first calculate
relative fitness of each diploid genotype. Relative fitness of a locus
is based on the selection coefficient for that locus $s_i\in \RR$,
which is a parameter of the simulation, usually between 0 and 1 in
empirical studies. Selection for a locus grows more important relative
to genetic drift as $s_i$ increases.
$$
\begin{array}{cccll}
\wij{BB} & \wij{BR} & \wij{RR} & \text{selection type} & \text{population color}\\
\hline
1 & 1+s_i/2 & 1+s_i & \text{positive}& \text{red}\\
1+s_i & 1+s_i/2 & 1 & \text{positive}& \text{blue}\\
1 & 1+s_i & 1 & \text{balancing}& \\
1 & 1 & 1 & \text{neutral}& 
\end{array}
$$

Then we update blue allele frequency for selection based on
Hardy-Weinberg equilibrium, which allows us to derive expressions for
genotype frequencies in terms of allele frequency:

$$\aij = \frac{
\wij{BB}\aijs^2 + \wij{BR}\aijs[1-\aijs]/2 }{
\wij{BB}\aijs^2 + \wij{BR}\aijs[1-\aijs] + \wij{RR}[1-\aijs]^2
   }$$

   We repeat the process for $t=2, ..., T$, and we take values
   $\alpha_{ij}(T)$ as the output allele frequencies of the
   simulation.



\subsection{The hierarchical bayesian Nicholson model}

To model the variation between observed allele frequencies in
different populations, the Nicholson model assigns a divergence
parameter $c_j$ to each population. The number of observed (blue)
alleles for locus $i$ in population $j$ is modeled as
$$x_{ij}\sim \Bin(N_{ij},\alpha_{ij})$$
where $\alpha_{ij}$ is the population allele frequency, and $N_{ij}$
is the total number of alleles (red or blue).

This quantity is in turned modeled by
$$\alpha_{ij}\sim N(\pi_i, c_j\pi_i(1-\pi_i))$$
a normal distribution truncated to the interval [0,1]. The
differentiation parameter $c_j$ is motivated by population genetics
\cite[section 2.2]{nicholson}.

The distribution of the hyperparameter for ancestral allele frequency
follows the prior distribution
$$\pi_i\sim \beta(a,a)$$
Values $a\in\{0.7,1\}$ were used, and showed similar results.

The population divergence hyperparameter follows the prior
distribution $$c_j\sim U[0,1]$$

The relationship between model parameters is more clearly summarised
in the following directed acyclic graph:

$$
\xymatrix{
  \pi_i \ar[rd] & & c_j \ar[ld] \\
  & \alpha_{ij} \ar[d] & \\
  & (x_{ij},N_{ij})
}
$$

The point of Bayesian statistics is to exploit Bayes' Rule to obtain
posterior distributions of the model parameters conditional on the
data. Generally, if use $x$ to signify the observed data and $\theta$
to denote the model parameters, we can write the posterior
distribution as
\begin{equation}
\label{bayes}
P(\theta|x) = \frac{P(x,\theta)}{P(x)} = 
\frac{P(x,\theta)}{\int P(x,\theta)d\theta} =
\frac{f(x|\theta)g(\theta)}{\int f(x|\theta)g(\theta)d\theta}
\end{equation}
where $f$ is the density of the data $x$ and $g$ is the density of the
model parameters $\theta$.

Since it is often difficult to evaluate the integral in the
denominator of equation \ref{bayes}, we instead turn to Markov Chain
Monte Carlo techniques to sample from the posterior
distribution. Essentially, we use a Metropolis-Hastings algorithm to
draw samples from a Markov chain, thus giving us an approximation of
the posterior distribution \cite{hastings}.

Thus, for each step in the chain $t$, we sample from the following
posterior distributions:
$$\alpha^t = P(\alpha|c^{t-1},\pi^{t-1},x)$$
$$\pi^t = P(\pi|c^{t-1},\alpha^t,a)$$
$$c^t = P(c|\pi^t,\alpha^t)$$

The model was first implemented using WinBUGS \cite{winbugs}. To
provide speed optimizations for the model fitting, a faster
model-fitting program was written in FORTRAN.

The posterior distribution for each model parameter was sampled 1000
times using MCMC, giving us an approximate posterior
distribution. However, to simplify the analyses, each distribution was
summarized using the mean. Thus when we refer to model parameter
estimates, we mean the sample mean of the values drawn from the
posterior distribution.

\subsection{PPP-value calculation theory}

The PPP-value for locus $i$ is defined as

$$\text{PPP}_i = 
P\left[T_i(y_{ij}^{\text{rep}},\theta)\geq T_i(y_{ij}^{\text{obs}},\theta)|y^{\text{obs}}\right]
$$
where $y_{ij}=x_{ij}/N_{ij}$ is the allele frequency for locus $i$ and
population $j$, $\theta$ is a vector of parameters, and $T_i$ is a
discrepancy criterion applied to replicated (rep) and observed (obs)
data sets.

We need to choose a discrepancy criterion which depends on both data
and parameters. Here, we use a $\chi^2$-type criterion:
$$T_i = \sum_{j=1}^P T_{ij}$$
with
$$T_{ij} = 
\frac{\left[y_{ij} - E(y_{ij}|\theta_{ij})\right]^2}{
  \operatorname{Var}(y_{ij}|\theta_{ij})}$$ where
$\theta_{ij}=(\pi_i,c_j,\sigma_{ij}^2)$ is a vector of parameters and
$\sigma_{ij}^2$ is $N_{ij}$ times the sampling variance of the observed
allele frequency given its true value $p_{ij}$ so that
$\sigma_{ij}^2=p_{ij}(1-p_{ij})$

We define the indicator variable
$$
I_i =
\begin{cases}
  1 & \text{if }T_i(y_{ij}^{\text{rep}},\theta)\geq T_i(y_{ij}^{\text{obs}},\theta) \\
  0 & \text{otherwise}
\end{cases}
$$

\subsection{PPP-value calculation implementation}

Calculation of PPP-values must be made in the context of sampling from
a stationary Markov chain. For each iteration $t$ through the chain,
we define this indicator value:

$$
p_{ij}^t =
\begin{cases} 
1 & \text{if }\left(
\frac{\operatorname{Bin}(N_{ij},\alpha_{ij}^t)}{N_{ij}} - \pi_i^t
\right)^2
>
\left(
\frac{Y_{ij}}{N_{ij}} - \pi_i^t
\right)^2\\
0 & \text{otherwise}
\end{cases}
$$
where $\operatorname{Bin}(\cdot,\cdot)$ represents a randomly generated
number from the binomial distribution.

Thus the PPP-value for locus $i$ is given by

$$
\text{PPP}_i = \frac 1 {PT} \sum_{j=1}^P \sum_{t=1}^T p_{ij}^t
$$
where $T$ is the number number of values sampled from the posterior
distribution and $P$ is the number of populations.
\section{Results}

\subsection{Simulation verification}

After obtaining allele frequencies from the simulator, we can do
diagnostic plots to visually verify that the allele frequencies are
evolving according to the theoretical evolution framework we had
envisioned. R packages lattice and ggplot2 are used to visualize these
multivariate data \cite{lattice,ggplot2}.

From population genetics, the expectation and variance of allele
frequency in a population under only genetic drift is given by:
$$E(\aij)=\pi_i$$
$$\operatorname{Var}(\aij)=\pi(1-\pi)\left[1-\left(1-\frac{1}{2N}\right)^{t-1}\right]$$
Thus we expect a good simulator of genetic drift to be unbiased for
the starting ancestral allele frequency, and to have variance
increasing with each generation of evolution $t$.

We visually checked how allele frequencies evolve over time by
examining lineplots of allele frequency over time
(\autoref{loci-over-time}). This plot shows 2 loci under positive
selection, 2 loci under balancing selection, and 2 loci not under
selection.

\fig{loci-over-time}{Allele frequency evolution of 6 loci in 12
  populations over 100 generations. Each panel represents a different
  locus, and each line therein represents a different
  population. Shown are two loci for balancing selection (left), no
  selection (middle), and positive selection (right).}

For the loci not under selection, the values of allele frequency rest
near the starting allele frequency, and the variance increases over
time. Thus, the loci not under selection exhibit the expected
characteristics and we can conclude the simulator works well for these
loci.

Similarly, for loci under selection, these plots reveal no signs of
departure from the hypotheses of our evolution simulator. That is,
high blue allele frequency is clearly favored for blue populations
under positive selection. For balancing selection, colored populations
evolve toward an allele frequency of 50\%, accurately simulating
selection of the heterozygote.

This plot was useful for looking at a few loci. However, to verify
that all loci behave according to expectations, we used dotplots of
final allele frequency (\autoref{fixation-endpoints}).

\fig{fixation-endpoints}{Final simulated blue allele frequency for
  1000 loci and 12 populations is shown in a dotplot. Loci are ordered
  on the horizontal axis by ancestral allele frequency, and then
  divided into 3 panels by selection state. Note that loci under
  selection display no signs of selective pressure when in neutral
  color populations. Inversely, all loci which are not under selection
  behave similarly, regardless of population color. Also, the symmetry
  between blue and red alleles is clearly visible.}

This dotplot clearly shows that all loci under selection display no
signs of selective pressure when in neutral color
populations. Inversely, all loci which are not under selection behave
similarly, regardless of population color. Also, the symmetry between
blue and red alleles is clearly visible. This dotplots efficiently
shows that the allele frequencies evolved according to the simulator
hypotheses.

\subsection{Model estimates}

We are simulating loci under selection, and analyzed them using the
pure-drift Nicholson model. Thus we expect that the loci under
selection will not fit the model well.

To diagnose dependence of model fit on selection state, we plot
ancestral allele frequency estimates for each loci versus actual
values from the simulation (\autoref{anc-est-plot}).

\fig{anc-est-plot}{Grouped scatterplots illustrate that model
  estimates of ancestral allele frequency are not robust to
  selection.}

This scatterplot clearly shows that neutral loci are well estimated by
the model, but loci under balancing and positive selection are not
well estimated. This result is sensible in view of the fact that the
Nicholson model was designed with only genetic drift in mind. Thus it
is expected that model parameters for loci under selection are not
estimated well.

To examine the robustness of these ancestral allele frequency
estimates to changes in number of populations and number of
generations, we did exhaustive simulations of several parameter
values. We did 9 simulations, fitting the model for each of them, with
25, 50, or 100 generations, and 4, 8, or 12 populations.

To display the results of the ancestral allele frequency estimates, we
used trellised scatterplots of estimates versus actual values
(\autoref{gen-pop}).

\fig{gen-pop}{Scatterplots of estimated versus actual ancestral allele
  frequency, trellised by number of generations and number of
  populations.}

From what we know about genetic drift, we expect that augmenting the
number of generations will increase the variance of the allele
frequencies, and thus increase the variance of the ancestral allele
frequency estimate. Similarly, we expect that increasing the number of
populations will give us more information about the ancestral allele
frequency, leading to more accurate estimates.

This series of scatterplots clearly shows that the estimates behave as
expected. Less accurate estimates are clearly seen with more
generations and fewer populations.

To evaluate if the model estimates of our FORTRAN program agree with
model estimates given by WinBUGS, we used both programs to fit the
model to a single simulation. Furthermore, Nicholson \etal\ note that
the model estimates are robust to changes in the prior distribution of
the $\pi_i$. We fit the Nicholson model to a single data set using
prior distributions of $\beta(0.7,0.7)$ and $\beta(1,1)=U[0,1]$.

The results of all these model fits are summarized in a scatter plot
matrix of ancestral allele frequency estimate versus actual value
(\autoref{notbeta}).

\fig{notbeta}{Scatter plot matrix for various values of
  ancestral allele frequency $\pi$ for a simulated data set (actual
  simulated values indicated by row/column simulated). Models using
  priors that follow a $\beta(1,1)$ (indicated by fortranr1 and
  winbugs1) and $\beta(0.7,0.7)$ (fortran.old, fortranr0.7, and
  winbugs0.7) were fit using WinBUGS and our FORTRAN program. For
  alleles under positive selection, there are small discrepancies
  between the FORTRAN and WinBUGS programs.}

The scatter plot matrix clearly shows that there are no significant
differences between models that use the same program. That is, the
choice of prior distribution of the ancestral allele frequencies has
little influence on model estimates.

However, there appears to be a difference between estimates from
WinBUGS and our FORTRAN program, uniquely for loci under positive
selection. Our FORTRAN parameter estimates are systematically higher
than the corresponding WinBUGS estimates for these loci. However,
since these loci are precisely the ones which do not fit the model, it
is expected that their estimates may differ.

To investigate the estimates of the population differentiation
parameters $c_j$, we first note that from population genetics we
expect that
\begin{equation}
  \label{c}
 c_j = 1 - (1 - \frac 1 N_j)^t\approx t/N_j  
\end{equation}
where $t$ is number of generations and $N_j$ is effective population
size of population $j$.

Note that this approximation implies that we expect estimates of $c_j$
to increase linearly over time. To visualize this linear trend, we
used lineplots of the estimate differentiation parameter $c_j$ over
time $t$ (\autoref{c-over-time-all}).

\fig[0.9]{c-over-time-all}{Lineplots of differentiation parameter
  estimates $c_j$ evolving over time. The model was fit for 4
  generations (50,100,150,200). Theoretical line shown in green is
  $t/N_j$.}

Note the linear behavior of the model estimates, as expected. However,
the slopes of the lines do not always match the expected theoretical
slopes, which can be attributed to approximation errors in
\autoref{c}.

We did simulations with uniform and variable population sizes to
determine if $c_j$ estimates were robust to population size
fluctuations. By comparing the panels in \autoref{c-over-time-all}, it
is evident that the model fits for population size 1000 fall in the
same range for both simulations. Thus we can conclude the model is
robust against population size variations.

\subsection{Characterization of loci fixation on model fit}

The simulation diagnostic dotplots also clearly show that the loci
under selection tend to get fixed at frequencies of 0 or 1 by the end
of the simulation (\autoref{fixation-endpoints}). These data seem to
violate the initial hypothesis that we are dealing with SNP data. That
is, SNP discovery is generally performed on small (<10) numbers of
individuals, thus data from SNP microarrays is necessarily biased to
favor loci which we have already observed are polymorphic in these
individuals. This phenomenon is called the ``ascertainment bias'' in
the literature.

To characterize if the model estimates are sensitive to the
ascertainment bias, we fit several models using non-fixed subsets of
the loci. The criteria used for calling a locus ``fixed'' are as
follows:
\begin{description}
\item[not.all.fixed] Throw out the locus if all subpopulations fixed
  (more stringent criterion; less loci will be ``fixed'').
\item[none.fixed] Throw out the locus if one or more subpopulations
  fixed (less stringent criterion; more loci will be ``fixed'').
\end{description}

To evaluate the effect of throwing out these loci on the total number
of loci left for input to the model, we made scatterplots of percent
of loci ``not fixed'' versus selection strength $s_i$
(\autoref{fixation-selection}). Essentially this told us that loci
under strong positive selection tend to be the ones which get called
``fixed'' and excluded from the model.

\fig{fixation-selection}{Percent of loci left after throwing out
  ``fixed'' loci, according to 2 criteria outlined in the text. Note
  how loci under strong positive selection are the loci which get
  excluded.}

To examine if there are any large differences between model estimates
when fixed data are not included, we made scatterplots of estimated
versus simulated ancestral allele frequency $\pi_i$ values
(\autoref{fixed}).

\fig{fixed}{Scatterplots of estimated and simulated ancestral allele
  frequency. The Nicholson model was fit for all loci, and 2 subsets
  of loci which were ``not fixed'' (see text).}

This plot indicates that the model fits are similar, regardless of the
number of loci included in the dataset. We also compare simulations
with large and small proportions of neutral loci. Note that the model
estimates behave similarly regardless of the number of loci included
in the model. Thus we conclude that the model estimates are robust
against the exclusion of fixed loci and the amount of neutral loci.

Thus we can conclude that no harm is done by leaving in the ``fixed''
loci, and we proceed with the rest of our analyses using all of the
simulated loci.

\subsection{Simulation summaries using animations}

To visualize 3 of the above simulation diagnostics at once, we made
combined plots of allele frequency time series, ancestral estimates,
and dotplots \autoref{sim-summary-plot}. This plot is more than just a
juxtaposition of the 3 individual plots, since we plot one locus over
time in the upper left, and then highlight this same locus in the
other 2 plots.

\fig{sim-summary-plot}{Frame 80/100 of a statistical animation that
  summarizes the evolution simulation. Note the time series plot for a
  single locus in the upper left. That same locus is highlighted with
  a circle in the upper right ancestral estimate plot, and with a
  vertical line in the bottom dotplot.}

To visualize the influence of the number of generations on each of
these diagnostic plots, we used to the animation
package \cite{animation} to create a series of plots, one for each
generation. These images are put together and viewed in sequence to
form a statistical animation that reveals the dependence on the number
of generations. The animations can be viewed on the accompanying
website:

\url{http://nicholsonppp.r-forge.r-project.org/}

The link between the plots, combined with the animation over several
generations, has proven to be a powerful pedagogical device that
encourages rapid understanding of the simulator and model
hypotheses. Multivariate statistical animations such as this can be
useful as teaching tools for students of statistical population
genetics.

\subsection{Prediction rates of the PPP-value classifier}

To evaluate the sensitivity and specificity of the PPP-value
classifier, we fit the model on 3 sets of 5 simulations with different
parameter values:

\begin{center}
\begin{tabular}{rrr}
  Set & Populations & Loci \\
  \hline
  usual & 12 & 1000 \\ %  
  few populations & 4 & 1000\\   % few
  many neutral loci & 12 & 19999   % neu
\end{tabular}
\end{center}

For each of the above parameter sets, we fixed constant parameter
values of population size 1000 and 100 generations of evolution. Then
we did 5 different simulations with 100 loci each of

$$
s_i\in\{0.001,0.01,0.032,0.1,1\}=\{10^{-3},10^{-2},10^{-1.5},10^{-1},10^0\}
$$
These values were chosen since 1 is a value higher than usually
observed in nature, and 0.001 is as weak as if there was no selection
at all.

For each of these sets we first made density plots of PPP-value
conditional on selection state for each $s_i$ value
(\autoref{dens-several-s}, \autoref{dens-several-s-few},
\autoref{dens-several-s-neu}).

\fig[0.8]{dens-several-s}{Density estimates for PPP-values of each
  selection state, given data sets simulated with different selection
  strengths $s_i$. Note how it gets easier to distinguish selection as
  the selection strength parameter increases.}

\fig[0.8]{dens-several-s-few}{Density estimates for PPP-values, for
  only 4 populations. With fewer populations it is more difficult to
  distinguish the behavior of loci under selection.}

\fig[0.8]{dens-several-s-neu}{Density estimates for PPP-values, when
  there is an abundance of neutral loci. In this case the densities
  are clearly distinguishable, but the sheer number of neutral loci
  makes a linear cutoff rule suboptimal.}

These density plots clearly visualize the different distributions of
PPP-values for different selection types. These plots suggest that low
PPP-values can be used to indicate positive selection, and high
PPP-values can be used to indicate balancing selection.

For the purposes of this work, we will limit ourselves to the
classification of a locus as positive or not positive, ignoring
balancing selection.

The density plots suggest the following classifier for positive loci:
$$\hat{S}_{i}(h)=\text{state of locus $i$ given threshhold $h$} = 
\begin{cases}
  \text{positive} & \text{if PPP}_i<h\\
  \text{none}     & \text{if PPP}_i<h
\end{cases}
$$
where $h$ is a PPP-value threshhold that will determine the false
positive/false negative rates.

These density plots also clearly show that increasing the value of the
selection coefficient $s_i$ tends to increase the separation of
distributions of PPP-values.

Examining \autoref{dens-several-s-few}, we see that with fewer
populations it is more difficult to distinguish the behavior of loci
under selection.

In \autoref{dens-several-s-neu}, we see that when there are very many
neutral loci the densities of the different selection states are
clearly distinguishable, but the sheer number of neutral loci makes a
linear cutoff rule suboptimal. For every possible threshhold, there
will be a large false positive rate.

Then we evaluated false positive and false negative rates for each
possible decision rule; that is, each possible cutoff for the
PPP-value. We used lineplots to trace the true positive, false
positive, false negative, and incorrect rates as a function of
classifier cutoff value (\autoref{cutoff-plot},
\autoref{cutoff-plot-few}, \autoref{cutoff-plot-neu}).

\fig[0.8]{cutoff-plot}{Lineplots of true positive, false positive, and
  false negative rates using the PPP-value classifier. Note the
  optimal cutoffs are near 0.35, according to empirical risk
  minimzation.}

\fig[0.8]{cutoff-plot-few}{Lineplots of true positive, false positive,
  and false negative rates using the PPP-value classifier on a data
  set with few populations. Note that optimal cutoffs are near 0.4,
  according to empirical risk minimization, but that only high
  selection values $s_i$ are detected. Best values for incorrect
  prediction are not as low as in the case where there are 12
  populations.}

\fig[0.8]{cutoff-plot-neu}{Lineplots of true positive, false positive,
  and false negative rates using the PPP-value classifier on a data
  set with many neutral loci. Note that with very many neutral
  alleles, the rate of false positives ascends very quickly. In this
  situation, the best cutoff value is around 0.2.}

These plots are used to investigate the best choice of threshhold for
the classifier in the simulations. We define the best threshhold $h^*$
as the one which minimizes the empirical risk:
$$
h^* = \stackrel{\text{argmin}}{h\in\{0,1\}} P(\hat{S}_{i}(h)\neq S_i)
$$
where $S_i$ is the actual selection state for locus $i$.

To identify the threshhold values of empiricical risk minimization, we
traced a horizontal grey line on the minimum value of the incorrect
curve, and identified the corresponding threshhold value with a
vertical grey line.

Receiver operating characteristics (ROCs) were also traced, to compare
all 15 simulations at the same time (\autoref{roc-desc},
\autoref{roc-s}). In ROCs, we plot lines of sensitivity against
$1-\text{specificity}$ for every possible threshhold value of the
classifier, where sensitivity is the true positive rate and
specificity is the false positive rate.

\fig{roc-desc}{ROCs for several selection strengths, neutral allele
  concentrations, and population numbers. As shown in the
  densityplots, increasing selection strengths $s_i$ tend to increase
  the area under the curve.}

\fig[0.8]{roc-s}{ROCs for several selection strengths, neutral allele
  concentrations, and population numbers. Note how the data sets with
  4 populations generate decision rules which are noticeably less
  powerful than those in the other 2 simulations.}

The ROCs in \autoref{roc-desc} show that increasing selection
strengths $s_i$ tend to increase the area under the curve, agreeing
with the densityplots.

In \autoref{roc-s}, note how the data sets with 4 populations generate
decision rules which are noticeably less powerful than those in the
other 2 simulations.

Additionally, ROCs were traced for 9 simulations comprising a cross of
3x3 parameter values: 25, 50, and 100 generations; 4, 8, 12
populations (\autoref{gen-pop-roc}).

\fig{gen-pop-roc}{ROCs show slight dependence of the PPP-value
  classifier on number of populations and generations. It is more
  difficult to accurately predict selection state for a smaller number
  of generations and populations.}

These ROCs suggest that it is more difficult to accurately predict
selection state for a smaller number of generations and populations.

\section{Conclusions and future work}

The Nicholson model investigated in this work has been thoroughly
tested by exhaustive simulation of genetic drift and selection. The
Nicholson model is a simple, robust, and useful Bayesian framework for
modeling population divergence under genetic drift from a recent
common ancestor.

To extend the Nicholson model to account for loci under selection, we
looked for loci which did not fit the Nicholson model well. Our
extension of the Nicholson model used PPP-values to classify the
selection state of each locus.

The PPP-values are an effective classifier for selection state when
the selection coefficient of the locus is sufficiently strong
($s_i>0.01$). The false positive rate of the classifier drops as the
selection strength of the loci increase. But if there are an
overwhelming majority of loci not under selection, there will
inevitably be high false positive rates.

Several graphical methods were used to visualize the data, including
use of statistical animations to understand the behavior of all loci
in the simulations. Such methods can be adapted as illustrative
teaching tools to facilitate rapid comprehension of these multivariate
data.

In this study, we did simulations and sensitivity analysis on every
possible threshhold to characterize the classifier. However, with real
data sets, we will need a more concrete criterion for choosing the
PPP-value threshhold for the classifier.

A hypothesis of the Nicholson model is that all loci are
independent. Genetic markers are found on linear chromosomes, so some
are closer than others. Some markers may even be in the same
gene. Thus the hypothesis of loci independence is clearly false. Some
model of correlation between the loci could be introduced into the
model.

The Nicholson model supposes that all populations diverge from their
common ancestor at the same time, which is false. Thus another
beneficial model complication would be to introduce some parameters
that model the more tree-like structure of real genetic histories.

To come up with a useful genome annotation method, we would need a
method of synthesizing classification of loci into classification of
genomic regions. Such methods for combining classifier predictions
exist, but would need to be adapted for this particular use with
PPP-values.

More work needs to be done to characterize the expected number of
false positives and false negatives in a real dataset. In particular,
a comparison of this classifier with other existing models of locus
selection state should be done.

The model we used only accounts for genetic drift, and detects
selection as aberrations from the model. An enhanced selection state
classifier could result if we introduced a more complicated Bayesian
model, with estimates of the selection state and/or coefficient.

Finally, we should apply this model to several well-characterized
empirical datasets. Due to the portability of the accompanying R
package, this should not be a difficult task. The results of our
simulations suggest that we will be able to accurately identify loci
under strong selection using this method.

\bibliographystyle{plain}
\bibliography{refs}

\end{document}
